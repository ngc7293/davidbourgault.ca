%----------------------------------------------------------------------------------------
%	PACKAGES AND OTHER DOCUMENT CONFIGURATIONS
%----------------------------------------------------------------------------------------

\documentclass[9pt]{developercv2} % Default font size, values from 8-12pt are recommended

%----------------------------------------------------------------------------------------

\begin{document}

%----------------------------------------------------------------------------------------
%	TITLE AND CONTACT INFORMATION
%----------------------------------------------------------------------------------------

\begin{minipage}[t]{0.45\textwidth}
	\vspace{-\baselineskip}
	\name{David}{Bourgault}

\end{minipage}
\begin{minipage}[t]{0.55\textwidth}
	\begin{minipage}[t]{0.5\textwidth}
		\vspace{-\baselineskip}
		\icon{MapMarker}{12}{Montréal}\\
		\icon{Phone}{12}{+1 418 208 9749}\\
		\icon{At}{12}{\href{mailto:contact@davidbourgault.ca}{contact@davidbourgault.ca}}\\
	\end{minipage}
	\begin{minipage}[t]{0.5\textwidth}
		\vspace{-\baselineskip}
		\icon{Github}{12}{\href{https://github.com/ngc7293}{github.com/ngc7293}}\\
		\icon{Linkedin}{12}{\href{https://www.linkedin.com/in/davidbourgault/}{davidbourgault}}\\
		\icon{Globe}{12}{\href{https://davidbourgault.ca}{davidbourgault.ca}}
	\end{minipage}
\end{minipage}

\cvsect{Sommaire}

Développeur logiciel avec plus de 3 ans d'expérience professionnelle en C, en C++ et en Python. Passionné par mon
métier, je suis à la recherche de nouveaux défis pour approfondir ma connaissance de systèmes connectés et haute
performance.

\cvsect{Expérience}
\begin{entrylist}
	\entry
	{2023/02 -- présent\\\emph{Temps-plein}}
	{Dévelopeur backend}
	{\href{https://ghgsat.com/fr/}{GHGSat Inc}}
	{
		\emph{
			GHGSat surveille les émissions de gaz à effets de serre grâce à sa constellation de satellites et de
			senseurs aéroportés. Les données recueillies sont offertes aux clients via le portail web Spectra.
		}

		\begin{itemize}
			\renewcommand{\labelitemi}{\raisebox{.45ex}{\rule{.6ex}{.6ex}}}
			\setlength\itemsep{-1pt}
			\item Développer des pipelines pour extraire, transformer et charger des données géoréférencées
			\item Participer à la transition vers AWS et Kubernetes
			\item Contribuer à un serveur web en Python avec Django
			\item Promouvoir des pratiques de code et de tests rigoureuses
		\end{itemize}

		\textbf{Technologies} : Python (3.8, 3.11), PostgreSQL (PostGIS), Panda, AWS, Kubernetes
	}
	\entry
	{2021/05 -- 2023/02\\\emph{Temps-plein}}
	{Développeur logiciel II}
	{\href{https://broadsign.com/}{Broadsign International LLC}}
	{
		\emph{
			La plateforme SaaS Broadsign Control offre une gestion intelligente des réseaux d'écrans et des contenus
			pour l'industrie de la publicité numérique "Out-of-home", ainsi que des algorithmes avancés de génération de
			listes de lecture.
		}

		\begin{itemize}
			\renewcommand{\labelitemi}{\raisebox{.45ex}{\rule{.6ex}{.6ex}}}
			\setlength\itemsep{-1pt}
			\item Contribuer à des applications serveur et de bureau en C++
			\item Contribuer à la modernisation du code source de la compagnie
			\item Concevoir, documenter et déployer des nouveaux systèmes logiciels
			\item Compiler et empaqueter des logiciels pour Windows et Linux
			\item Contribuer à une infrastructure de tests automatisés
		\end{itemize}

		\textbf{Technologies} : C++ (11, 14, 17), Qt, gRPC, Kubernetes, Microservices, PostgreSQL, GTest
	}
	\entry
	{2020/08 -- 2019/12\\\emph{Stage}}
	{Développeur logiciel}
	{\href{https://broadsign.com/}{Broadsign International LLC}}
	{
		\emph{
			La plateforme SaaS Broadsign Control offre une gestion intelligente des réseaux d'écrans et des contenus
			pour l'industrie de la publicité numérique "Out-of-home", ainsi que des algorithmes avancés de génération de
			listes de lecture.
		}

		\begin{itemize}
			\renewcommand{\labelitemi}{\raisebox{.45ex}{\rule{.6ex}{.6ex}}}
			\setlength\itemsep{-1pt}
			\item Contribuer à des applications serveur et de bureau en C++
			\item Déboguer le code applicatif et déveloper de nouvelles fonctionnalités
			\item Contribuer à une infrastructure de tests automatisés
		\end{itemize}

		\textbf{Technologies} : C++ (11, 14, 17), Qt, gRPC, Kubernetes, Microservices, PostgreSQL, GTest
	}
	\entry
	{2018/09 -- 2018/12\\\emph{Stage}}
	{Dévelopeur embarqué}
	{\href{https://www.silabs.com/}{Silicon Laboratories Inc.}}
	{
		\emph{
			Silicon Labs conçoit des microcontrôlleurs et solutions SoC pour lespériphériques connectés.
		}

		\begin{itemize}
			\renewcommand{\labelitemi}{\raisebox{.45ex}{\rule{.6ex}{.6ex}}}
			\setlength\itemsep{-1pt}
			\item Contribuer des correctifs et des améliorations au kernel en temps-réel \emph{Micrium}
			\item Étendre une infrastructure de profilage automatique en Python
			\item Déboguer le code applicatif et système embarqué en C
		\end{itemize}

		\textbf{Technologies} : C embarqué, Python, RTOS (Micrium), JTAG
	}
\end{entrylist}

\cvsect{Éducation}
\begin{entrylist}
	\setlength\itemsep{-1pt}
	\entry
	{2017/05 -- 2021/04}
	{Baccalauréat en Génie Logiciel}
	{\href{https://etsmtl.ca}{École de Technologie Supérieure}}
	{
		\vspace{-14pt}
		\begin{itemize}
			\renewcommand{\labelitemi}{\raisebox{.45ex}{\rule{.6ex}{.6ex}}}
			\setlength\itemsep{-1pt}
			\item Mention Génie+ pour implication dans la vie étudiante
			\item Spécialisation en systèmes embarqués
		\end{itemize}
	}
\end{entrylist}

\cvsect{Projets}
\begin{entrylist}
	\entry
	{2017/05 -- 2021/05\\{\small\emph{Developer}}}
	{\href{https://clubrockets.ca/}{RockÉTS -- Fusées haute puissance}}
		{\href{https://clubrockets.ca/}{École de Technologie Supérieure}}
	{
		\vspace{-14pt}
		\begin{itemize}
			\renewcommand{\labelitemi}{\raisebox{.45ex}{\rule{.6ex}{.6ex}}}
			\setlength\itemsep{-1pt}
			\item Lancer 5 fusées sondes à des altitudes de 10 000 et 30 000 pieds
			\item Gérer le projet de rearchitecture de l'avionique 2019
			\item Programmer les systèmes embarqués temps-réel et la télémétrie en C et C++
		\end{itemize}
	}
	\entry
	{2021/01 -- now\\{\small\emph{Developer}}}
	{\href{https://github.com/ngc7293/ganymede}{Ganymède (Project de fin d'études)}}
	{École de Technologie Supérieure}
	{
	\vspace{-14pt}
	\begin{itemize}
		\renewcommand{\labelitemi}{\raisebox{.45ex}{\rule{.6ex}{.6ex}}}
		\setlength\itemsep{-1pt}
		\item Concevoir une plateforme SaaS pour l'hydroponie automatisée en C++
		\item Implémenter la communication GRPC/Protobuf sur une plateforme embarquée
	\end{itemize}
	\vspace{-12pt}
	}
\end{entrylist}
\vspace{-\baselineskip}

\cvsect{Compétences}
\begin{minipage}[t]{0.5\textwidth}
	\textbf{Langages}: C++, C, Python, Typescript, Rust

	\textbf{Cadriciels}: Qt, React, Django, Flask

	\textbf{APIs}: REST, GraphQL, GRPC
\end{minipage}
\begin{minipage}[t]{0.5\textwidth}
	\textbf{Environnements}: Windows, Linux, STM32, ESP32

	\textbf{Base de données}: PostgreSQL, Redis, MongoDB

	\textbf{Nuages}: Docker, Kubernetes
\end{minipage}

\cvsect{À propos}
\textbf{Langues}: Français/Anglais (bilingue), Allemand (niveau A2)

\textbf{Intérêts}: Photographie, Technologies aérospatiales, Anthropologie
\end{document}
